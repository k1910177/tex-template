%% Math
\usepackage{physics}
\usepackage{mathtools}
\usepackage{resizegather}

%% cref
%% http://tug.ctan.org/tex-archive/macros/latex/contrib/cleveref/cleveref.pdf
\usepackage{cleveref} 
\crefname{enumi}{}{}
\crefname{equation}{式}{式}
\crefname{figure}{図}{図}
\crefname{table}{表}{表}
\crefname{algorithm}{アルゴリズム}{アルゴリズム}
\crefformat{section}{#2#1#3節}
\crefformat{subsection}{#2#1#3項}
\newcommand{\crefpairconjunction}{と}
\newcommand{\crefrangeconjunction}{から}
\newcommand{\crefmiddleconjunction}{、}
\newcommand{\creflastconjunction}{、および}
\let\normalref\ref
\renewcommand{\ref}{\cref}


%% Figure
% \usepackage{graphicx}
% \usepackage{lmodern}
% 並列実行するとsvg-inkscapeディレクトリが競合するのでinkscapepathをjobname付きで指定
% \usepackage[
%   inkscapelatex=false,
%   inkscapepath=./svg-inkscape-\jobname,
%   inkscapeopt=--export-text-to-path,
% ]{svg}
% \usepackage{pdfpages}
% \usepackage{here}
% \usepackage{subcaption}
% \usepackage{multirow}
% \usepackage{color}
% \usepackage{colortbl} % tabularのセルに色つける
% \usepackage{makecell} % \thead でセル内改行ができる
% \usepackage{tablefootnote} % tabular内でfootnoteを作れる
% \usepackage{siunitx} % 小数点揃え
% \usepackage{tikz}
% \usetikzlibrary{intersections, calc, arrows, positioning, arrows.meta}
% \usepackage{bytefield}
% \usepackage{tabularx}
% \newcolumntype{R}{>{\raggedleft\arraybackslash}X} % 右揃え
% \newcolumntype{C}{>{\centering\arraybackslash}X} % 中央揃え
% \usepackage{arydshln}
% \usepackage{booktabs}

%% Cryptocode
%% https://github.com/arnomi/cryptocode
\usepackage[
  lambda,
  operators,
  advantage,
  sets,
  adversary,
  landau,
  probability,
  notions,
  logic,
  ff,
  mm,
  primitives,
  events,
  complexity,
  asymptotics,
  keys
]{cryptocode}

%% Others
\usepackage{xurl}
% \usepackage{url}
\usepackage{bm} % 記号も太字
% \usepackage{emoji}
\usepackage{enumitem} % itemlabelなどの詳細設定
% \usepackage{comment}
% \usepackage{subfiles}
% \usepackage{todonotes}
% \setuptodonotes{backgroundcolor=red!25,bordercolor=red,linecolor=red}
% \setlength{\marginparwidth}{2cm} % todonotesが収まるように
