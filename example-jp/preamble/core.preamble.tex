% LuaTeX-ja の使い方 https://ja.osdn.net/projects/luatex-ja/wiki/LuaTeX-ja%E3%81%AE%E4%BD%BF%E3%81%84%E6%96%B9

% CreationDate, ModDate, Trailer IDを削除して再現可能なPDFを作成できる
% https://www.pragma-ade.com/general/manuals/luatex.pdf
\pdfvariable suppressoptionalinfo 608


%% lualatexの設定
\usepackage{luatexja}
\usepackage{luatexja-fontspec}
\usepackage[
  % ipaex,
  haranoaji,
]{luatexja-preset} % フォント指定

%% pdfの設定
\usepackage[
  hidelinks, % リンクに枠線を付けない
  setpagesize=false, % hyperrefでページサイズを変更しない
  pdfversion=2.0,
  pdfa, % pdf/a準拠
  pdfpagemode=UseNone, % PDFビューワでサイドバーなどを自動で開かない
  pdfcreator={}, % LaTeX with hyperrefを埋め込まない
]{hyperref} % pdfの目次、メタデータ
% \usepackage{pdflscape} % pdfのページを横向きにする


%% unicode設定
\usepackage{newunicodechar}
% Replace "、" with ","
\newunicodechar{、}{,}
% Replace "。" with "."
\newunicodechar{。}{.}


%% Environment
\usepackage{amsmath, amsthm, amssymb}
\theoremstyle{definition}
\newtheorem{definition}{定義}
\newtheorem{theorem}{定理}
\newtheorem{lemma}{補題}
\newtheorem{corollary}{系}
\newtheorem{remark}{注意}
\renewcommand{\proofname}{\textbf{証明}}


%% Reference(以下の\usepackage{biblatex}のどちらかを選ぶ)
%% 引用時の表示が[著者イニシャル+年]
%% 日本語論文の引用時,[岩渡22]と変な感じになってしまうため,[IW22]の形にするためにbibファイルに以下の処理を加える:bibファイル冒頭に「@preamble{ " \newcommand{\noop}[1]{} " }」を追記し,和文論文の各著者の前に\noopコマンドを使ってイニシャルを明記しておく.例:bibファイルで「渡邉洋平」となっているところを「\noop{W}渡邉洋平」とする.
%% 引用時の表示が[番号]
\usepackage[
  backend=biber, 
  style=numeric-comp, 
  sorting=ynt, 
  sortcites=true, 
  backref=false, 
  date=year, 
  maxnames=100, 
  minalphanames=3, 
  maxalphanames=3, 
  isbn=false, 
  doi=false, 
  url=false,
  defernumbers=true,
]{biblatex}
\DeclareFieldFormat[online]{date}{\thefield{year}年\thefield{month}月\thefield{day}日 閲覧}
%% 以下は表示のカスタマイズ
\DeclareFieldFormat*{title}{\mkbibquote{#1\adddot}}	%% タイトルの後にピリオドが来るように
\renewcommand\multicitedelim{\addcomma\space}	%% 複数引用時に [A; B; C] となるところを [A, B, C] に変更
\AtEveryBibitem{
  \iffieldequalstr{langid}{japanese}
    {\DeclareDelimFormat{finalnamedelim}{\addcomma\space}}
    {\DeclareDelimFormat{finalnamedelim}{\addcomma\space\bibstring{and}\space}}
}	%% 日本語論文(langind={japanese}のbibitem)の最後の著者の手前の and を削除し,和文論文引用時の違和感を排除


%% Macro
\newcommand{\two}{\mathrm{I}\hspace{-1.2pt}\mathrm{I}}
\newcommand{\todo}[1]{\textcolor{blue}{\textbf{$\langle \! \langle$To Do: #1$\rangle \! \rangle$}}}
\newcommand{\iwmt}[1]{\textcolor{green}{ $\langle \! \langle$岩本: ``#1"$\rangle \! \rangle$}}
\newcommand{\wtnb}[1]{\textcolor{magenta}{ $\langle \! \langle$渡邉: ``#1"$\rangle \! \rangle$}}
